\documentclass[12pt]{article}
\usepackage[utf8]{inputenc}
\usepackage{float}
\usepackage{amsmath}


\usepackage[hmargin=3cm,vmargin=6.0cm]{geometry}
%\topmargin=0cm
\topmargin=-2cm
\addtolength{\textheight}{6.5cm}
\addtolength{\textwidth}{2.0cm}
%\setlength{\leftmargin}{-5cm}
\setlength{\oddsidemargin}{0.0cm}
\setlength{\evensidemargin}{0.0cm}



\begin{document}

\section*{Student Information } 
%Write your full name and id number between the colon and newline
%Put one empty space character after colon and before newline
Full Name : Mustafa Ozan ALPAY \\
Id Number : 2309615 \\

% Write your answers below the section tags
\section*{Answer 1}
\subsection*{a} 
We can say that 
\begin{itemize}
    \item the set of possible green candy counts are $\{ 0, 2, 4 \} $
    \item the set of possible red candy counts are $\{ 4, 5, 6, 7, 8, 9, 10 \} $
    \item the set of possible blue candy counts are $\{ 1, 3, 5, 7, 9 \} $
\end{itemize}
The generating functions for the aforementioned possible selections are as follows;
\begin{itemize}
    \item $(x^0 + x^2 + x^4)$ 
    \item $(x^4 + x^5 + x^6 + x^7 + x^8 + x^9 + x^{10})$
    \item $(x^1 + x^3 + x^5 + x^7 + x^9)$
\end{itemize}
Since the total sum must be equal to $10$, in the product of the generating functions, the coefficient of the term with $x^{10}$ would give us the number of ways.
$$ = (x^0 + x^2 + x^4) \cdot (x^4 + x^5 + x^6 + x^7 + x^8 + x^9 + x^{10}) \cdot (x^1 + x^3 + x^5 + x^7 + x^9) $$
$$ = x^{23} + \dotsb + 9x^{11} +
6x^{10} + 6x^{9} + \dotsb + x^{5}  $$
Since the coefficient of the term with $x^{10}$ is 6, there are 6 ways to select 10 candies in such ways.
                
\subsection*{b}

We can say that 
\begin{itemize}
    \item the set of possible green candy counts are $\{ 0, 2, 4 \} $
    \item the set of possible red candy counts are $\{ 4, 5 \} $
    \item the set of possible blue candy counts are $\{ 1, 3, 5 \} $
\end{itemize}
The generating functions for the aforementioned possible selections are as follows;
\begin{itemize}
    \item $(x^0 + x^2 + x^4)$ 
    \item $(x^4 + x^5)$
    \item $(x^1 + x^3 + x^5)$
\end{itemize}
Since the total sum must be equal to $10$, in the product of the generating functions, the coefficient of the term with $x^{10}$ would give us the number of ways.
$$ = (x^0 + x^2 + x^4) \cdot (x^4 + x^5) \cdot (x^1 + x^3 + x^5) $$
$$ = x^{14} + x^{13} + 2x^{12} + 2x^{11} +
3x^{10} + 3x^{9} + 2x^{8} + 2x^{7} + x^6 + x^5  $$
Since the coefficient of the term with $x^{10}$ is 3, there are 3 ways to select 10 candies in such ways.

\subsection*{c}
Let us take an arbitrary $G(x)$ where $G(x) = \dfrac{1}{(1-2x)(1+3x)}$. By applying the partial fractions method, we can get the following,
$$ G(x) = \dfrac{1}{5} \cdot \left( \dfrac{3}{1+3x}+\dfrac{2}{1-2x} \right)$$ 
Since $F(x) = 7x^2 \cdot G(x)$, we can say that,
$$ F(x) = \dfrac{7}{5} \cdot x^2 \cdot \left( \dfrac{3}{1+3x}+\dfrac{2}{1-2x} \right)$$ 
By using the following equation, 
$$ \dfrac{1}{1-ax} = \sum\limits_{k=0}^{\infty} a^k x^k $$ 
we can rewrite $F(x)$ as the following,
$$ F(x) = \dfrac{7}{5} \cdot x^2 \cdot \left( 3 \cdot \sum\limits_{k=0}^{\infty} -3^k x^k + 2 \cdot \sum\limits_{k=0}^{\infty} 2^k x^k \right) $$ 
By rearranging the terms,
$$ F(x) = \dfrac{7}{5} \cdot \left( 3 \cdot \sum\limits_{k=2}^{\infty} (-3^{k-2} \cdot x^{k}) + 2 \cdot \sum\limits_{k=2}^{\infty} (2^{k-2} \cdot x^{k} ) \right) $$ 
Since the upper and the lower limits of the summations are the same, we can gather them together.
$$ F(x) = \dfrac{7}{5} \cdot  \sum\limits_{k=2}^{\infty} ( -3^{k-1} \cdot x^{k} + 2^{k-1} \cdot x^{k} ) $$
$$ F(x) = \dfrac{7}{5} \cdot  \sum\limits_{k=2}^{\infty} \left( x^{k} \cdot (-3^{k-1} + 2^{k-1}) \right) $$


\subsection*{d}
First, let's try to find the value of $s_0$. By putting $n = 1$ in the recurrence, we get, 
$$ s_1 = 8s_0 + 10^0$$
Since we know that $s_1$ is $9$, we can conclude that $s_0$ is 1. \\ 
\\
If we multiply both sides of the recurrence relation by $x^n$, we obtain, 
$$ s_n x^n = 8s_{n-1} x^n + 10^{n-1} x^n $$ 
Let $G(x)$ be the generating function of the sequence $s_0, s_1, s_2, \dotsb$ with the following equation,
$$ G(x) = \sum\limits_{n=0}^{\infty} s_n x^n$$ 
By summing both sides of the multiplied recurrence relation starting with $n=1$, we get,

$$ G(x) - 1 = \sum\limits_{n=1}^{\infty} s_n x^n = \sum\limits_{n=1}^{\infty} ( 8s_{n-1} x^n + 10^{n-1} x^n ) $$
$$  = 8 \sum\limits_{n=1}^{\infty} s_{n-1} x^n + \sum\limits_{n=1}^{\infty} 10^{n-1} x^n  $$ 
$$  = 8x \sum\limits_{n=1}^{\infty} s_{n-1} x^{n-1} + x \sum\limits_{n=1}^{\infty} 10^{n-1} x^{n-1}  $$ 
$$  = 8x \sum\limits_{n=1}^{\infty} s_{n-1} x^{n-1} + x \sum\limits_{n=1}^{\infty} 10^{n-1} x^{n-1}  $$ 
$$  = 8x \sum\limits_{n=0}^{\infty} s_{n} x^{n} + x \sum\limits_{n=0}^{\infty} (10x)^{n}  $$ 
$$ = 8x \cdot G(x) + \dfrac{x}{1-10x} $$ 
$$ G(x) - 1 = 8x \cdot G(x) + \dfrac{x}{1-10x} $$ 
By solving the equation for $G(x)$, we get, 
$$ G(x) = \dfrac{1-9x}{(1-8x)(1-10x)} $$ 
By using partial fractions, we get, 
$$ G(x) = \dfrac{1}{2} \cdot \left( \dfrac{1}{1-8x} + \dfrac{1}{1-10x} \right) $$ 
By using the following equation (for $ |ax| < 1$) ,
$$  \sum\limits_{n=0}^{\infty} (ax)^{n} = \dfrac{1}{1-ax} $$ 
$$ G(x) = \dfrac{1}{2} \cdot \left(  \sum\limits_{n=0}^{\infty} (8x)^{n} +  \sum\limits_{n=0}^{\infty} (10x)^{n} \right)$$ 
$$ = \sum\limits_{n=0}^{\infty} \dfrac{1}{2} \left( (8x)^{n} +  (10x)^{n} \right) $$ 
$$ = \sum\limits_{n=0}^{\infty} \dfrac{1}{2} (8^n + 10^n)x^{n} $$ 
Therefore, we can conclude that 
$$ s_n = \dfrac{1}{2} \left( 8^n+10^n \right) $$ 


\section*{Answer 2}
\subsection*{a}
\begin{itemize}
    \item Let us pick $n=20$, $m=4$, and $k=8$.
    \item $C_{20}=\{4,6,8,9,10,12,14,15,16,18,20\}$
    \item $A_4=\{4,8,12,16,20\}$
    \item $A_8=\{8,16\}$
    \item Since $4 | 8$, $A_8 \subseteq A_4$ must hold. 
    \item Since $\{8,16\} \subseteq \{4,8,12,16,20\}$ is a valid statement, we can say that we verified that if $m|k$, then $A_k \subseteq A_m$.
\end{itemize}
\subsection*{b}
\begin{itemize}
    \item We need to prove that the prime roots of a composite number is always less than or equal to the square-root of the number.
    \begin{itemize}
        \item Since $n$ is a positive composite number, we can rewrite $n$ as $x \cdot y$, where $x,y \in Z$ and $1 < x, y < n$. 
        \item Let us suppose $x \leq y$. 
        \item In the case that $x > \sqrt{n}$, $y \geq x > \sqrt{n}$ also must hold.
        \item However, if $y \geq x > \sqrt{n}$ is a correct statement, then $n = xy > \sqrt{n} \cdot \sqrt{n} > n$, which is a contradiction. Therefore our claim is incorrect.
        \item $x \leq \sqrt{n}$ is a correct claim. 
        \item In that case, the $x$ value can be a composite or a prime number, which does not affect the correctness of ``the prime roots of a composite number is always less than or equal to the square-root of the number".
    \end{itemize}
    \item Since the $A_i$ sets of non-prime numbers will be the subsets of their prime roots' sets, the union of the prime $A_i$ sets will be equal to the union of all $A_i$ sets. 
    \item Since we already proved that the ``prime roots of a composite number is always less than or equal to the square-root of the number", we can conclude that the following equation is correct:
    $$ \bigcup\limits_{i=2}^{n-1} A_i = \bigcup\limits_{primes\: p \leq \sqrt{n}} A_p $$ 
\end{itemize}
\subsection*{c}
\begin{itemize}
    \item To explain this phenomena better, let us visualize it by giving $n=45$, and $m=6$. In that case, $ A_6 = \{12,18,24,30,36,42\}$. We can clearly see that $\vert A_6 \vert = 6$. (Which also can be calculated by $\vert A_6 \vert = \left\lfloor \dfrac{45}{6} \right\rfloor - 1 = 6$
    \item In any number $n$, there are $\left\lfloor \dfrac{n}{m} \right\rfloor$ times $m$ exist. Since while counting the $A_m$ count we ignore the first $n$, $\vert A_m \vert = \left\lfloor \dfrac{n}{m} \right\rfloor - 1$ must hold for any $m \geq 2$.
\end{itemize}
\subsection*{d}
\begin{itemize}
    \item To understand this question easily, let us visualize it with picking $a=3, b=4, n=30$.
    \item Since $3$ and $4$ are relatively prime, and since they both are less than $30$, the first statement holds.
    \item $A_3=\{6,9,12,15,18,21,24,27,30\}$
    \item $A_4=\{8,12,16,20,24,28\}$
    \item $A_{12}=\{24\}$
    \item $(A_3 \cap A_4) - A_{12} = {12}$
    \item From the example above we can see that the only element existing in the $(A_a \cap A_b) - A_{ab}$ is the $ab$ element itself; since due to the definition of $A$, $ab$ will exist in both $A_a$ and $A_b$, but it will not exist in $A_{ab}$. 
\end{itemize}
\subsection*{e}
\begin{itemize}
    \item We can generalize the $A_p$ sets as the following;
    $$ A_p=\{x | x = p \cdot n, n \in N, n > 1 \}$$
    \item Since for each $p \in P$, the $p$ values will be relatively prime as well, the only common elements in the sets will be their least common multiple, and its multiples.
    \item For relatively prime numbers, the least common multiple will be the multiplication of the numbers. Therefore, the least common multiple value of every $p \in P$ is basically the multiplication of each number. 
    \item The least common multiple can be denoted as the following;
    $$ lcm(\forall p \in P) = \prod\limits_{p \in P} p $$ 
    \item To find the cardinality of this set, we can simply apply the floor function. Therefore we can formulate it as follows;
    $$ \left\vert \bigcap\limits_{p \in P} A_p \right\vert = \left\lfloor \frac{n}{\prod\limits_{p \in P} p} \right\rfloor $$ 
\end{itemize}
\subsection*{f}
\begin{itemize}
    \item Let us write $C_{45}, A_2, A_3, $ and $A_5$. 
    \item $ C_{45}=\{4,6,8,9,10,12,14,15,16,18,20,21,22,24,25,26,27,28,30,32,33,34,35, 36,38, $ \\ $39,40,42,44,45\} $
    \item $A_2=\{4,6,8,10,12,14,16,18,20,22,24,26,28,30,32,34,36,38,40,42,44\}$
    \item $A_3=\{6,9,12,15,18,21,24,27,30,33,36,39,42,45\}$
    \item $A_5=\{10,15,20,25,30,35,40,45\}$
    \item From the sets above, we can see that each set has common elements with each other. By using the Inclusion-Exclusion Principle, we can say the following;
    $$ \vert C_{45} \vert = \vert A_2 \vert + \vert A_3 \vert + \vert A_5 \vert - \vert A_2 \cap A_3 \vert - \vert A_2 \cap A_5 \vert - \vert A_3 \cap A_5 \vert + \vert A_2 \cap A_3 \cap A_5 \vert  $$ 
\end{itemize}
\subsection*{g}
\begin{itemize}
    \item Since $\vert C_{45} \vert = 30$, by the Inclusion-Exclusion Principle, the total number of primes up to 45 should be,
    $$ = 45 - \vert C_{45} \vert - 1$$ 
    \item Since there are 45 numbers between $[1,45]$, subtracting $C_{45}$ from this would give us the number of primes $+1$ (since we included 1 in our counting). Therefore we must subtract 1 from this calculation, which yields to 14 prime numbers.
\end{itemize}

\section*{Answer 3}
\subsection*{a}
\begin{itemize}
    \item Let us pick the ordered sets $x = (a,b)$, $y = (c,d)$, and $z = (e,f)$; where $\{a,b,c,d,e,f\} \in Z$. By the definition of \textit{transitivity}, if $(x,y) \in Z$ and $(y,z) \in Z$, then $(x,z) \in Z$ for all $x,y,z \in Z$. 
    \item For $x$ and $y$, we can write the following logical statement:  $$(a,b) \ll (c,d) \leftrightarrow  (a < c) \lor ((a=c) \land (b \leq d))$$
    \item For $y$ and $z$, we can write the following logical statement:  $$(c,d) \ll (e,f) \leftrightarrow  (c < e) \lor ((c=e) \land (d \leq f))$$
    \item If the logical statement is a \textit{tautology} for $x$ and $y$, and if the logical statement is also a \textit{tautology} for $y$ and $z$, we can merge the two statements if the $\ll$ relation is transitive.
    $$(a,b) \ll (c,d) \ll (e,f) \leftrightarrow (((a < c) \lor ((a=c) \land (b \leq d))) \land ((c < e) \lor ((c=e) \land (d \leq f)))) $$
    \item Let us denote 
    \begin{enumerate}
        \item $p \equiv a < c$
        \item $q \equiv c < e$
        \item $r \equiv (a = c) \land (b \leq d)$
        \item $s \equiv (c = e) \land (d \leq f)$
    \end{enumerate}
    \item The right hand side of the logical statement becomes as the following;
    $$ (p \lor r) \land (q \lor s) $$ 
    \item By using \textbf{Distributive laws}, we get 
    $$(p \lor r) \land (q \lor s) \equiv  (p \land q) \lor (p \land s) \lor (r \land q) \lor (r \land s)$$ 
    \item If we rewrite the $p, q, r, s$ values into the statement back, we get, 
    \begin{enumerate}
        \item $p \land q \equiv (a < c) \land (c < e) \equiv a < e$
        \item $p \land s \equiv (a < c) \land ((c = e) \land (d \leq f)) \equiv (a < c = e) \land (d \leq f)$
        \item $r \land q \equiv ((a = c) \land (b \leq d)) \land (c < e) \equiv (a = c < e) \land (b \leq d)$
        \item $r \land s \equiv ((a = c) \land (b \leq d)) \land ((c = e) \land (d \leq f)) \equiv (a = c = e) \land (b \leq d \leq f) $
    \end{enumerate}
    \item If the given $\ll$ relation is \textit{transitive}, it should hold if $(a < e) \lor ((a = e) \land (b \leq f))$
    \item Since we are interested in the \textit{transitive} property, the $2^{nd}$ and $3^{rd}$ statements do not say much since they have terms with $c$ and $d$. In order to continue with our investigation, we are going to take a look at the statements with $a,b,e,f$. 
    \begin{enumerate}
        \item $p \land q \equiv a < e$
        \item $p \land s  \equiv a < e$ \textit{(since the original statement had a $\land$ relation, both sides of the statement must hold.)}
        \item $r \land q \equiv a < e$ \textit{(since the original statement had a $\land$ relation, both sides of the statement must hold.)}
        \item $r \land s \equiv (a = e) \land (b \leq f) $
    \end{enumerate}
    \item Since the initial statement had 4 different logical statements conjoined using a logical or statement, in the case when only one of them is true, the given statement would yield a True value. By conjoining the aforementioned 4 statements, we get;
    $$(a < e) \lor ((a = e) \land (b \leq f))$$
    \item Since we reached our initial statement, the given relation is \textbf{transitive} by logic rules.
\end{itemize}

\subsection*{b}
\begin{itemize}
    \item To show that the $\propto$ relation is an equivalence relation, we need to show that it is reflexive, symmetric, and transitive. 
    \begin{enumerate}
        \item \textbf{Reflexivity}
        \begin{itemize}
            \item Let us pick a real number that satisfies $x \geq k$ for every $x$.
            \item We can select any $k$ and the equations $f(x)=f(x)$ and $g(x)=g(x)$ will be satisfied.
            \item If $f(x)=g(x)$ for $\propto$ relation, then we can say that the $\propto$ relation is reflexive for any $x$ that is $x \geq k$.
        \end{itemize}
        \item \textbf{Symmetricity}
        \begin{itemize}
            \item Let us pick a real number that satisfies $x \geq k$ for every $x$.
            \item We can select any $k$ and the equation $f(x)=g(x)$ will be satisfied for the relation $f \propto g$.
            \item Similarly, if $f(x)=g(x)$, then $g(x)=f(x)$ must hold.
            \item Therefore, we can say that for any $x \geq k$, the relation $g \propto f$ will be satisfied.
            \item Since both $f \propto g$ and $g \propto f$ holds for any $k$ that satisfies $x \geq k$, we can conclude that the $\propto$ relation is symmetric for any $x$ that is $x \geq k$.
        \end{itemize}
        \item \textbf{Transitivity}
        \begin{itemize}
            \item Let us pick 3 functions $f, g, h$; respectively.
            \item To prove transivity, we need to show that if $f \propto g$ and $g \propto h$, then $f \propto h$ must hold for any $x$ that is $x \geq k$.
            \item For an arbitrary $k$ such that $x \geq k$, $f(x)=g(x)$ and $g(x)=h(x)$ holds by the definition of the relation.
            \item By this equalities, we can say that $f(x)=h(x)$ for any $x$ that is $x \geq k$.
            \item Therefore we can say that the given relation is transitive.
        \end{itemize}
    \end{enumerate}
    \item Since we have shown that the given $\propto$ relation is \textbf{reflexive}, \textbf{symmetric}, and \textbf{transitive}, we can conclude that $\propto$ is an \textbf{equivalence relation}.
\end{itemize}




\end{document}

​

